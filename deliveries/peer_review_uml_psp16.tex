\documentclass[12pt]{report}
\usepackage[utf8]{inputenc}
\usepackage[T1]{fontenc}
\usepackage[italian]{babel}

\begin{document}
	\author{Lorenzo Motelli, Giovanni Moretti, Manuel Ianieri}
	\title{\textbf{Peer Review Gruppo PSP16}}
	\maketitle
	
	\section*{Cose positive}
		\begin{itemize}
			\item UML ben strutturato, con già la divisione in package.
			\item Utilizzare la classe Map di java può risultare molto utile per identificare univocamente ogni professore e studente, ma anche complesso essendo una classe che viene utilizzata soprattutto per controlli su liste.
			\item Le classi sono ben strutturate, con pochi metodi essenziali.
			\item Buon implementazione generale dei pattern.
			\item Buono che è stato messo controller.
		\end{itemize}
	\section*{Cose negative}
		\begin{itemize}
			\item Gestione personaggi non particolarmente chiara in questo UML iniziale.
			\item Ripetizione delle isole, GameTable e MotherNature hanno, presumibilmente, entrambi la stessa coppia di lista di isole presenti nel gioco.
			\item La DinignRoom ha sempre lo stesso numero di studenti massimi che può contenere, a questo punto dichiarare direttamente che sia una variabile di tipo final.
			\item A cosa servono i due metodi privati checkProfessorOwenership e getTowerZone?
			\item Perché solo un'enumerazione è stata collegata? E quelle non collegate sono enumerazioni vuote? A cosa servono? STATUS in particolare.
			\item Come viene implementato ExpertGame?
		\end{itemize}
	\newpage
	\section*{Confronto e conclusioni}
		La struttura generale del model è molto simile, con varie classi che gestiscono in ugual modo determinate parti (ad esempio Table è GameBoard, GeneralGame è Game). Le principali differenze sono sulla scelta di gestione delle diverse componenti del gioco, come gli studenti utilizzando liste o map, oppure madre natura come boolean o classe a sé stante.
		In conclusione per essere una versione iniziale, l'UML risulta dettagliato e ragionato. Per questo motivo saltano all'occhio problematiche di comprensione e/o di dettaglio. Con un confronto diretto molte di queste incomprensioni si sarebbero potute chiarire immediatamente, ma leggendo soltanto l'UML alcune scelte non sono state comprese. Nel complesso un buon lavoro.
\end{document}